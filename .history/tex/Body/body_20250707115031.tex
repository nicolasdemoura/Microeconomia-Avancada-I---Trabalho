This study investigates the causal effect of rainfall on labor supply decisions in the Sao Paulo Metropolitan Region (RMSP), Brazil. Specifically, we examine whether precipitation affects both the probability of working and commute time for work-related activities. Weather conditions may influence work decisions through multiple channels: transportation costs and delays, direct utility effects from weather preferences, and productivity considerations. Understanding these relationships has important implications for labor market policy and urban planning, particularly in large metropolitan areas where weather-related disruptions can have significant economic consequences. Our analysis contributes to the growing literature on environmental factors and labor market outcomes by providing evidence from one of the largest urban agglomerations in the developing world.

This research relates to several strands of economic literature. First, it builds on the extensive literature examining labor supply responses to various shocks and constraints, including transportation costs \citep{zenou2009urban}, weather conditions \citep{connolly2008here}, and urban characteristics \citep{moretti2011local}. Second, our work contributes to the emerging field of environmental economics that studies how weather and climate affect economic outcomes \citep{hsiang2016climate}. Previous studies have found mixed evidence on weather effects on labor supply: some find negative effects of extreme weather on work attendance \citep{lee2016temperature}, while others document positive effects through reduced leisure alternatives \citep{connolly2008here}. Third, this study adds to the literature on developing country labor markets, where informal employment and flexible work arrangements may lead to different responses to weather shocks compared to developed economies. Our focus on Brazil is particularly relevant given its large informal sector and the importance of weather-sensitive outdoor work.

Our rainfall data comes from the Centro Nacional de Monitoramento e Alertas de Desastres Naturais (CEMADEN), Brazil's national center for monitoring and alerting natural disasters. CEMADEN operates a comprehensive network of automatic weather stations throughout Brazil, providing high-frequency precipitation measurements. For the RMSP, we utilize daily rainfall data from 2017, which covers the period of our labor market survey. The CEMADEN data offers several advantages: it provides precise temporal coverage matching our survey period, has wide spatial coverage across the metropolitan region, and uses standardized measurement protocols ensuring data quality. We aggregate daily rainfall measurements to match the survey reference periods, creating variables for total rainfall (in millimeters) and a binary indicator for any precipitation. This approach allows us to capture both the extensive margin (occurrence of rainfall) and intensive margin (amount of rainfall) effects on labor supply decisions.

Our primary labor market data comes from the 2017 Origin-Destination Survey (Pesquisa Origem-Destino) conducted by the Sao Paulo Metropolitan Transportation Company (Companhia do Metropolitano de Sao Paulo). This comprehensive household survey collects detailed information on travel patterns, work activities, and demographic characteristics for residents of the RMSP. The survey covers approximately 32,000 households and provides rich information on individual work decisions, commute times in minutes, and socioeconomic characteristics. Key variables include binary indicators for work participation on the survey day, duration of work-related commutes in minutes, and comprehensive individual controls such as age, gender, education level, employment status, occupation, economic sector, and employment type. The survey's design allows us to observe both whether individuals worked on specific days and how long they spent commuting to work-related activities.

To combine the rainfall and labor market data, we developed a spatial matching procedure that links survey respondents to weather station measurements. We first calculated the geographic centroids of each origin-destination zone in the survey using municipal boundaries and zone definitions. Next, we created Voronoi polygons around each CEMADEN weather station, which partition the metropolitan area such that each location is assigned to its nearest weather station. We then matched each survey respondent's zone centroid to the corresponding weather station, assigning them the rainfall measurements from their nearest station. This approach ensures that each individual receives rainfall data from the most geographically relevant weather station while maintaining the spatial precision necessary for causal identification. The matching process accounts for the irregular spatial distribution of weather stations and survey zones, providing a robust link between environmental conditions and individual labor market outcomes.

Our empirical strategy employs a reduced-form approach that exploits plausibly exogenous variation in daily rainfall to identify causal effects on labor supply. We estimate the following models for work probability and commute time:

\begin{align*}
Y_{iz} &= \alpha + \beta_1 \text{Rain}_{iz} + \beta_2 \text{Rainfall}_{iz} + \mathbf{X}_{iz}'\gamma + \delta_z + u_{iz}
\end{align*}

where $Y_{iz}$ represents either a binary work indicator or commute time for individual $i$ in zone $z$, $\text{Rain}_{iz}$ is a dummy for any precipitation, $\text{Rainfall}_{iz}$ is total rainfall in millimeters, $\mathbf{X}_{iz}$ includes individual controls, and $\delta_z$ represents zone fixed effects. We estimate five specifications with progressively more controls: rainfall dummy only, rainfall amount only, both rainfall variables, both plus individual controls, and the full specification with zone fixed effects. The identifying assumption is that daily rainfall variation is exogenous to individual labor supply decisions, which is plausible given the short-term nature of weather variation and the difficulty of perfectly predicting daily precipitation.

Our empirical analysis reveals statistically insignificant effects of rainfall on both work participation and commute time in the RMSP. Across all specifications, we find no evidence that rainfall affects the probability of working or the duration of work-related commutes. The point estimates for both the rainfall dummy and continuous rainfall variables are close to zero and lack statistical significance at conventional levels. For work participation, the estimated effects suggest that rainfall has virtually no impact on work decisions among employed individuals. Similarly, for commute time, we find no meaningful relationship between precipitation and travel time among those who do work. These null results are robust across different model specifications, including those with comprehensive individual controls and zone fixed effects. The precision of our estimates allows us to rule out economically meaningful effects, suggesting that daily rainfall variation does not substantially influence short-term labor supply decisions in this urban Brazilian context. These findings contrast with some previous studies but may reflect the particular characteristics of the RMSP labor market, including the prevalence of indoor work, well-developed transportation infrastructure, and the flexibility of work arrangements in urban Brazil.

\begin{figure}[H]
    \centering
    \includegraphics[width=0.8\textwidth]{../figures/rainfall_histogram.png}
    \includegraphics[width=0.8\textwidth]{../figures/trips_histogram.png}
    \includegraphics[width=0.8\textwidth]{../figures/duration_histogram.png}
    \caption{Histograms showing the distributions of rainfall (top), number of trips (middle), and commute durations (bottom) in the dataset.}
    \label{fig:hist}
\end{figure}

\begin{figure}[H]
    \centering
    \includegraphics[width=0.8\textwidth]{../figures/rmsp_base.png}
    \includegraphics[width=0.8\textwidth]{../figures/rmsp_voronoi.png}
    \caption{Spatial representation of the RMSP region: (top) zone partitions, (bottom) Voronoi diagram illustrating spatial partitions based on centroid proximity to weather stations.}
    \label{fig:rmsp_voronoi}
\end{figure}

\input{../output/panel_a}
\input{../output/panel_b}