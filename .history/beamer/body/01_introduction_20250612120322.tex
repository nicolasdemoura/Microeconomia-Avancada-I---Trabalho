

\section{Introdução}
\begin{frame}
    \frametitle{Introdução}

    \begin{itemize}
        \item O efeito da chuva na assiduidade no trabalho é um fenômeno observado em diversas regiões do mundo.
        \item A chuva pode impactar a capacidade dos trabalhadores de chegar ao trabalho, afetando sua produtividade e, consequentemente, a economia local.
        \item Este estudo busca analisar como a variação na precipitação influencia a frequência dos trabalhadores em seus postos de trabalho.
    \end{itemize}
\end{frame}

\section{Dados}
\begin{frame}
    \frametitle{Dados}
    Utilizarei dois conjuntos de dados principais:
    \begin{itemize}
        \item \textbf{Pesquisa de Origem e Destino (OD)}, que contém informações sobre os deslocamentos dos trabalhadores na Região Metropolitana de São Paulo em 2017.
        \item \textbf{Centro Nacional de Monitoramento e Alertas de Desastres Naturais (CEMADEN)}, que fornece dados horários de precipitação para a mesma região.
    \end{itemize}
\end{frame}


\begin{frame}
    \frametitle{Dados}
    \framesubtitle{OD}
    \begin{itemize}
        \item
    \end{itemize}
\end{frame}


\begin{frame}
    \frametitle{Dados}
    \framesubtitle{CEMADEN}
    \begin{itemize}
        \item
    \end{itemize}
\end{frame}

\section{Metodologia}[plain]
\begin{frame}
    \frametitle{Metodologia}
    \begin{align*}
    y_{ist} &= \tau \cdot \text{P}_{st} + X_{ist}'\beta + \alpha_{s} + \veps_{ist} \\ 
    \end{align*}
    onde:
    \begin{itemize}
        \item $y_{ist}$ é a variável dependente, que representa se o trabalhador $i$ na zona $s$ no tempo $t$ foi ao trabalho.
        \item $\text{P}_{st}$ é a precipitação na zona de origem $s$ no tempo $t$.
        \item $X_{ist}$ são as variáveis de controle, como idade, gênero, escolaridade do trabalhador.
        \item $\alpha_{s}$ são os efeitos fixos da zona.
        \item $\veps_{ist}$ é o termo de erro.
    \end{itemize}
\end{frame}

\begin{frame}
    \frametitle{Metodologia}
    \begin{itemize}
        \item Nosso parâmetro de interesse é $\tau$, que mede o efeito da precipitação na probabilidade de um trabalhador ir ao trabalho.
        \item Esperamos que $\tau$ seja negativo, indicando que um aumento na precipitação reduz a probabilidade de assiduidade no trabalho.
        \item Potencialmente buscaremos identificar efeitos heterogêneos, considerando trabalhadores formais e informais, bem como atuantes em setores distintos.
    \end{itemize}
\end{frame}