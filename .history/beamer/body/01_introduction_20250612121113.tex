

\section{Introdução}
\begin{frame}
    \frametitle{Introdução}
    \begin{itemize}
    \item Mudanças climáticas têm aumentado a frequência de eventos extremos.
    \item Chuva intensa pode dificultar o deslocamento de trabalhadores.
    \item Esse tipo de fricção afeta decisões de oferta de trabalho e produtividade.
    \item Quantificar esses efeitos é importante para políticas urbanas e de mobilidade.
    \end{itemize}
\end{frame}

\section{Dados}
\begin{frame}
    \frametitle{Dados}
    Utilizarei dois conjuntos de dados principais:
    \begin{itemize}
        \item \textbf{Pesquisa de Origem e Destino (OD)}, que contém informações sobre os deslocamentos dos trabalhadores na Região Metropolitana de São Paulo em 2017.
        \item \textbf{Centro Nacional de Monitoramento e Alertas de Desastres Naturais (CEMADEN)}, que fornece dados horários de precipitação para a mesma região.
    \end{itemize}
\end{frame}


\begin{frame}
    \frametitle{Dados}
    \framesubtitle{OD}
    \begin{itemize}
        \item A pesquisa foi realizada em 2017, e é representativa da Região Metropolitana de São Paulo.
        \item Permite observar deslocamentos casa-trabalho e motivos de viagem.
        \item Contém microdados individuais com idade, sexo, renda, ocupação e setor.
    \item Permite criar variáveis indicando se a pessoa foi ao trabalho no dia da pesquisa.
    \item Informações espaciais sobre zonas de origem e destino permite georefereciar os dados.
    \end{itemize}
\end{frame}

\begin{frame}[plain]
    \frametitle{Dados}
    \framesubtitle{OD}
    \begin{figure}[H]
        \centering
        \includegraphics[width=0.66\textwidth]{../figures/RMSP_centroids.png}
        \caption{Centroides das zonas de origem e destino da Pesquisa de Origem e Destino (OD) na Região Metropolitana de São Paulo.}
        \label{fig:od_centroids}
    \end{figure} 
\end{frame}



\begin{frame}
    \frametitle{Dados}
    \framesubtitle{CEMADEN}
    \begin{itemize}
    \item Dados de chuva em alta frequência por estação pluviométrica.
    \item Construção de série diária de precipitação por estação.
    \item Mapeamento da estação mais próxima a cada zona da OD.
    \item Geração de polígonos de Voronoi para definir áreas de cobertura de cada estação.
    \end{itemize}
\end{frame}

\section{Metodologia}
\begin{frame}[plain]
    \frametitle{Metodologia}
    \begin{align*}
    y_{ist} &= \tau \cdot \text{P}_{st} + X_{ist}'\beta + \alpha_{s} + \veps_{ist} \\ 
    \end{align*}
    onde:
    \begin{itemize}
        \item $y_{ist}$ é a variável dependente, que representa se o trabalhador $i$ na zona $s$ no tempo $t$ foi ao trabalho.
        \item $\text{P}_{st}$ é a precipitação na zona de origem $s$ no tempo $t$.
        \item $X_{ist}$ são as variáveis de controle, como idade, gênero, escolaridade do trabalhador.
        \item $\alpha_{s}$ são os efeitos fixos da zona.
        \item $\veps_{ist}$ é o termo de erro.
    \end{itemize}
\end{frame}

\begin{frame}
    \frametitle{Metodologia}
    \begin{itemize}
        \item Nosso parâmetro de interesse é $\tau$, que mede o efeito da precipitação na probabilidade de um trabalhador ir ao trabalho.
        \item Esperamos que $\tau$ seja negativo, indicando que um aumento na precipitação reduz a probabilidade de assiduidade no trabalho.
        \item Potencialmente buscaremos identificar efeitos heterogêneos, considerando trabalhadores formais e informais, bem como atuantes em setores distintos.
    \end{itemize}
\end{frame}