\begin{frame}
    \frametitle{Spatiotemporal Settings}
    \begin{itemize}
        \item Spatiotemporal settings are common in many fields, such as economics, epidemiology, and environmental science.
        \item Consider a setting with $N$ units observed over $T$ time periods:
        \begin{align}
        Y_{it} &= \beta\cdot X_{it} + U_{it}
        \end{align}
        where we are interested in testing the null hypothesis $H_0: \beta = 0$.
        \item However, often it is not plausible to assume that the errors $u_{it}$ are independent across units and time periods.
    \end{itemize}
\end{frame}

\begin{frame}
    \frametitle{Model- vs. Design-Based Approaches}
    \begin{itemize}
        \item Diferent approaches \citep{abadie2020ECTA} have been proposed to deal with spatiotemporal correlation in inference:
        \begin{itemize}
           \item \textbf{Model-based approaches:} asymptotically valid under assumptions about the covariance structure.
           \item \textbf{Design-based approaches:} small-sample valid under homogeneous treatment effects and stationarity assumptions.
        \end{itemize}
        \item We propose to combine both approaches to obtain better inference in spatiotemporal settings.
    \end{itemize}    
\end{frame}